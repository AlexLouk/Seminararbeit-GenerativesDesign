\begin{abstract}
    Die Entwicklung von generativen KI-Modellen hat in den letzten Jahren erheblichen Einfluss auf die Arbeit von Designern genommen. In dieser Seminararbeit werden wir untersuchen, wie generatives KI-Design die Art und Weise verändert hat, wie Designer ihre Arbeit erledigen. Wir werden die verschiedenen Anwendungen von generativem KI-Design untersuchen, wie zum Beispiel die Generierung von Logos, Schriftarten und Layouts.
    Des Weiteren werden wir die Vorteile und Herausforderungen von generativem KI-Design untersuchen. Einerseits können Designer durch die Verwendung von generativem KI-Design zeitsparender und effektiver arbeiten. Andererseits kann es jedoch auch dazu führen, dass Designer weniger kreativ und innovativ arbeiten, da sie sich auf die von der KI erstellten Designs verlassen.
    Wir werden auch die Rolle von generativem KI-Design in der Zukunft des Designs betrachten und diskutieren, wie Designer und KI-Modelle in Zukunft zusammenarbeiten können. Schließlich werden wir auch die ethischen und rechtlichen Aspekte von generativem KI-Design betrachten und diskutieren, wie man sicherstellen kann, dass es nicht zu unerwünschten Auswirkungen auf die Gesellschaft kommt.  
    Insgesamt wird diese Seminararbeit eine umfassende Analyse des Einflusses von generativem KI-Design auf die heutige Arbeit von Designern liefern und ein Verständnis dafür vermitteln, wie diese Technologie die Zukunft des Designs beeinflussen wird.   
\end{abstract}