\begin{abstract}
    Die Entwicklung von generativen Modellen hat in den letzten Jahren erheblichen Einfluss auf die Arbeit in der Designbranche genommen. In dieser Seminararbeit werden wir untersuchen, wie generatives Design die Art und Weise verändert hat, wie Designer ihre Arbeit erledigen. Wir werden die verschiedenen Anwendungen von generativem Design in Bereichen wie Produktdesign und Architektur untersuchen. Außerdem werden wir die Herausforderungen erläutern und die Vor- und Nachteile gegeneinander abwägen. 
    
    Einerseits können Designer durch die Verwendung von generativem Design zeitsparender und effektiver arbeiten. Andererseits kann es jedoch auch dazu führen, dass Designer weniger kreativ und innovativ arbeiten, da sie sich auf die von der KI erstellten Designs verlassen.
    Wir werden auch die Rolle von generativem Design in der Zukunft des Designs betrachten und diskutieren, wie Designer und Modelle in zusammenarbeiten können. Schließlich werden wir auch die ethischen und rechtlichen Aspekte von generativem Design betrachten und diskutieren, wie man sicherstellen kann, dass es nicht zu unerwünschten Auswirkungen auf die Gesellschaft kommt.  
    Insgesamt wird diese Seminararbeit eine umfassende Analyse des Einflusses von generativem Design auf die heutige Arbeit von Designern liefern und ein Verständnis dafür vermitteln, wie diese Technologie die Zukunft des Designs beeinflussen wird.   


\subsection*{abstractOptionZwei}
Abstract:

Diese Seminararbeit untersucht das Thema des generativen Designs und seine Auswirkungen auf die kreativen Gestaltungsprozesse in der Designbranche. Der Fokus liegt auf der Definition und den Konzepten des generativen Designs sowie dem historischen Überblick über seine Entwicklung. Darüber hinaus werden verschiedene Methoden des generativen Designs wie parametrisches Design, algorithmisches Design, evolutionäre Algorithmen und datengesteuertes Design vorgestellt. Es werden auch die Anwendungen des generativen Designs in Bereichen wie Architektur, Produktgestaltung, Grafikdesign, Modedesign, Industriedesign, Medizin und Gesundheitswesen untersucht. Die Herausforderungen und Zukunftsaussichten des generativen Designs werden ebenfalls diskutiert, einschließlich ethischer und rechtlicher Aspekte sowie technologischer Entwicklungen. Schließlich werden die Ergebnisse dieser Arbeit zusammengefasst, die Forschungsfrage beantwortet und eine kritische Bewertung sowie ein Ausblick auf die zukünftige Bedeutung des generativen Designs in der Designbranche gegeben. Diese Arbeit trägt dazu bei, das Verständnis und die Wertschätzung des generativen Designs als innovativen Ansatz für kreative Gestaltungsprozesse zu vertiefen.

\end{abstract}