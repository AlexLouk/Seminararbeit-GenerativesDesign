\subsection*{Problemstellung}

Die Designbranche steht vor der Herausforderung, effiziente Gestaltungsprozesse zu finden und innovative Lösungen für komplexe Probleme zu entwickeln. Traditionelle Designansätze stoßen jedoch häufig an ihre Grenzen, da sie auf manuellen Prozessen und subjektiver Intuition basieren. Dies führt zu begrenzten Möglichkeiten der Variation und Exploration sowie zu einem erhöhten Zeitaufwand für die Entwicklung von Designs.

Um diesen Herausforderungen zu begegnen, wird das generative Design als vielversprechender Ansatz betrachtet. Es nutzt algorithmische Methoden, parametrische Modelle und maschinelles Lernen, um kreative Lösungen automatisch zu generieren. Durch die Integration computergestützter Prozesse und automatisierter Generierung eröffnet das generative Design neue Möglichkeiten jenseits traditioneller Designmethoden.

Die Problemstellung dieser Arbeit besteht darin, das Potenzial des generativen Designs vollständig zu erfassen und seine Auswirkungen auf die kreativen Gestaltungsprozesse in der Designbranche zu verstehen. Dabei sollen Fragen beantwortet werden wie: Wie beeinflusst das generative Design die kreative Intuition und den Entscheidungsprozess von Designern? Wie können die generierten Designs bewertet und optimiert werden, um den Anforderungen der Nutzer gerecht zu werden? Welche ethischen und rechtlichen Aspekte sind im Zusammenhang mit dem generativen Design zu berücksichtigen?

Ziel dieser Seminararbeit ist es, ein umfassendes Verständnis für das generative Design zu entwickeln und seine Auswirkungen auf die Designbranche zu untersuchen. Durch die Beantwortung dieser Fragen sollen neue Perspektiven und Erkenntnisse gewonnen werden.

\subsection*{Zielsetzung}

Diese Seminararbeit zielt darauf ab, den Einfluss des generativen Designs auf kreative Gestaltungsprozesse in der Designbranche zu untersuchen. Dabei werden die grundlegenden Konzepte und Methoden des generativen Designs erläutert, um ein umfassendes Verständnis für diese innovative Designpraxis zu vermitteln. Darüber hinaus werden konkrete Anwendungen des generativen Designs in verschiedenen Bereichen wie Architektur, Produktgestaltung, Grafikdesign, Kunst, Modedesign, Industriedesign sowie Medizin und Gesundheitswesen untersucht.

Die Arbeit setzt sich auch mit den Herausforderungen des generativen Designs auseinander und bietet einen Ausblick auf zukünftige Entwicklungen und potenzielle Innovationen. Dabei werden ethische und rechtliche Aspekte im Zusammenhang mit generativem Design berücksichtigt.

Durch umfassende Literaturrecherche und Analyse soll die Forschungsfrage beantwortet werden: "Wie beeinflusst generatives Design die kreativen Gestaltungsprozesse in der Designbranche?" Die Auswirkungen des generativen Designs auf die Kreativität und den Gestaltungsprozess werden untersucht und kritisch bewertet.

Die Ergebnisse dieser Arbeit sollen dazu beitragen, ein besseres Verständnis für die Möglichkeiten und Herausforderungen des generativen Designs in der Designbranche zu gewinnen und einen Beitrag zur Diskussion über die Zukunft kreativer Gestaltungsprozesse zu leisten.

\subsection{Aufbau der Arbeit}

Der Aufbau der Arbeit folgt einer klaren Struktur, die es dem Leser ermöglicht, die Entwicklung des Themas nachzuvollziehen. Nach einer einführenden Einleitung werden in Kapitel II die Grundlagen des generativen Designs erläutert, um ein solides Fundament für das weitere Verständnis zu schaffen. Kapitel III widmet sich den verschiedenen Methoden des generativen Designs und gibt einen Überblick über ihre Funktionsweise.

Kapitel IV beschäftigt sich mit den konkreten Anwendungen des generativen Designs in verschiedenen Bereichen. Dabei werden für jeden Bereich Fallbeispiele präsentiert, um die praktische Anwendung zu veranschaulichen. In Kapitel V werden die Herausforderungen und zukünftigen Perspektiven des generativen Designs diskutiert, wobei ethische, rechtliche und technologische Aspekte berücksichtigt werden.

Abschließend erfolgt in Kapitel VI eine Zusammenfassung der Ergebnisse, die Beantwortung der Forschungsfrage sowie eine kritische Bewertung und ein Ausblick auf zukünftige Entwicklungen. Die Arbeit wird mit einem Literaturverzeichnis abgeschlossen, das die verwendeten Quellen angibt.