\subsection*{Problemstellung}
Mit generativem Design hat künstliche Intelligenz  in den letzten Jahren einen großen Einfluss auf  Produktdesign und Technologie gehabt 
 Produktherstellung hat gewonnen. Dies ermöglicht es Unternehmen, ihren Kunden innovative und maßgeschneiderte Lösungen anzubieten 
 Automatische Generierung und Optimierung verschiedener Designoptionen. Gleichzeitig eine Reihe von Parametern 
 und die Kriterien, die das Design beeinflussen, werden definiert, und dann schafft künstliche Intelligenz eine Vielzahl von Designmöglichkeiten 
 die Anforderungen erfüllen. Anschließend können die besten Optionen ausgewählt werden, um das Endprodukt zu entwickeln.  Dieser Prozess ist eine effektive Möglichkeit, gleichzeitig die Produkteffizienz und -effektivität zu verbessern  
 reduziert den Materialverbrauch und die Produktionskosten. Es hat sich gezeigt, dass Unternehmen generatives Design nutzen 
 und mithilfe künstlicher Intelligenz  ihre Produkte schneller auf den Markt bringen, wettbewerbsfähiger  und besser sein 
 Kundenzufriedenheit erreichen. 
  Ein großartiges Beispiel für den Einsatz von generativem Design mit künstlicher Intelligenz sind die Nike Flyprint-Schuhe. 
 Nike hat in Zusammenarbeit mit Autodesk ein Designer-Tool zum Entwerfen von Schuhen mithilfe generativen Designs entwickelt. Der 
 Die Schuhe sind speziell für Sportler konzipiert und sollen optimale Passform und Leistung bieten. verwenden 
 Durch generatives Design mit künstlicher Intelligenz konnte Nike schnell und effizient Tausende von Designmöglichkeiten generieren und 
 Wählen Sie die besten Optionen für Schuhe. Das Ergebnis war ein innovativer Schuh, der den Bedürfnissen von Sportlern gerecht wird 
  und gleichzeitig wird der Materialverbrauch reduziert. 
 
 Ein Beispiel hierfür sind die generativ mit künstlicher Intelligenz gestalteten Flyprint-Schuhe von Nike. 
 viele Anwendungen  künstlicher Intelligenz im Produktdesign. Aber wie genau ist generatives Design mit KI? 
 wird verwendet und welche Auswirkungen hat es auf die Produktdesignbranche?

 \subsection*{Zielsetzung}
 Der Zweck dieser Seminararbeit besteht darin, die Auswirkungen des generativen Designs auf kreative Designprozesse in der Designbranche zu untersuchen. Um ein umfassendes Verständnis dieser innovativen Designpraxis zu erlangen, sollen die grundlegenden Konzepte und Methoden des generativen Designs erläutert werden. Darüber hinaus werden spezifische Anwendungen des generativen Designs in verschiedenen Bereichen wie Architektur, Produktdesign, Grafikdesign, Kunst, Modedesign, Industriedesign sowie Medizin und Gesundheitswesen untersucht. 
 
 Das Papier diskutiert auch die Herausforderungen des reproduktiven Designs und gibt einen Überblick über die weitere Entwicklung und mögliche Innovationen. Berücksichtigt werden ethische und rechtliche Aspekte im Zusammenhang mit reproduktivem Design. Die durch eine umfassende Literaturrecherche und -analyse zu beantwortende Forschungsfrage lautet: „Wie wirkt sich generatives Design auf kreative Designprozesse in der Designbranche aus?“ Der Einfluss generativen Designs auf die Kreativität und den Designprozess wird untersucht und kritisch bewertet. 
 
 Die Ergebnisse dieser Arbeit sollen zu einem besseren Verständnis der Chancen und Herausforderungen des generativen Designs in der Designbranche beitragen und die Diskussion über die Zukunft kreativer Designprozesse anregen.

 \subsection*{Aufbau der Arbeit}
 Der Aufbau des Werkes folgt einer logischen Struktur, die es dem Leser ermöglicht, die thematische Entwicklung  nachzuvollziehen. Nach der Einführung erläutert Kapitel II die Grundlagen des generativen Designs, um eine solide Grundlage für das weitere Verständnis zu schaffen. Kapitel III widmet sich  verschiedenen Methoden des generativen Designs und gibt einen Überblick über deren Funktionsweise. 
 In Kapitel IV werden spezifische Anwendungen des generativen Designs in verschiedenen Bereichen erörtert, und Fallstudien aus jedem Bereich werden vorgestellt, um die praktische Anwendung zu veranschaulichen.  Kapitel V diskutiert die Herausforderungen und Zukunftsperspektiven des reproduktiven Designs unter Berücksichtigung ethischer, rechtlicher und technischer Aspekte. Abschließend erfolgt in Kapitel VI eine Zusammenfassung der Ergebnisse, Antworten auf die  Forschungsfrage sowie eine kritische Einschätzung und Sicht auf die weitere Entwicklung. Die Arbeit endet mit einer Liste der verwendeten Quellen.



