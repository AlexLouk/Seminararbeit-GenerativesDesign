\subsection*{Problemstellung}
Die Designbranche steht vor der Herausforderung, kreative Gestaltungsprozesse zu optimieren und innovative Lösungen für komplexe Probleme zu finden. Traditionelle Designansätze stoßen jedoch oft an ihre Grenzen, da sie auf manuellen Prozessen und subjektiver Intuition basieren. Dies führt zu begrenzten Möglichkeiten der Variation und Explorationsfreiheit sowie zu einem erhöhten Zeitaufwand für die Entwicklung von Designs.

Um diese Herausforderungen zu bewältigen, wird das generative Design als vielversprechender Ansatz angesehen. Es nutzt algorithmische Methoden, parametrische Modelle und maschinelles Lernen, um kreative Lösungen automatisch zu generieren. Durch die Integration von rechnergestützten Prozessen und automatisierter Generierung eröffnet das generative Design neue Wege der Gestaltung, die über traditionelle Designmethoden hinausgehen.

Die Problemstellung besteht darin, das Potenzial des generativen Designs voll auszuschöpfen und zu verstehen, wie es die kreativen Gestaltungsprozesse in der Designbranche beeinflusst. Welche Auswirkungen hat das generative Design auf die kreative Intuition und den Entscheidungsprozess der Designer? Wie können die generierten Designs bewertet und optimiert werden, um den Bedürfnissen der Nutzer gerecht zu werden? Wie können ethische und rechtliche Aspekte im Zusammenhang mit dem generativen Design berücksichtigt werden?

Diese Problemstellung bildet den Ausgangspunkt für diese Seminararbeit, um ein umfassendes Verständnis für das generative Design zu entwickeln und seine Auswirkungen auf die Designbranche zu untersuchen. Durch die Beantwortung dieser Fragen können neue Perspektiven und Erkenntnisse 


\subsection*{Zielsetzung}
Die vorliegende Seminararbeit hat zum Ziel, den Einfluss des generativen Designs auf kreative Gestaltungsprozesse in der Designbranche zu untersuchen. Dabei sollen die grundlegenden Konzepte und Methoden des generativen Designs erläutert werden, um ein umfassendes Verständnis für diese innovative Designpraxis zu vermitteln. Zudem sollen konkrete Anwendungen des generativen Designs in verschiedenen Bereichen wie Architektur, Produktgestaltung, Grafikdesign, Kunst, Modedesign, Industriedesign sowie Medizin und Gesundheitswesen untersucht werden. 

Die Arbeit befasst sich ebenfalls mit den Herausforderungen, denen das generative Design gegenübersteht, und bietet einen Ausblick auf zukünftige Entwicklungen und potenzielle Innovationen. Dabei werden ethische und rechtliche Aspekte im Zusammenhang mit generativem Design berücksichtigt. 

Durch eine umfassende Literaturrecherche und Analyse soll die Forschungsfrage beantwortet werden: "Wie beeinflusst generatives Design die kreativen Gestaltungsprozesse in der Designbranche?" Dabei werden die Auswirkungen von generativem Design auf die Kreativität und den Gestaltungsprozess untersucht und kritisch bewertet. 

Die Ergebnisse dieser Arbeit sollen dazu beitragen, ein besseres Verständnis für die Möglichkeiten und Herausforderungen des generativen Designs in der Designbranche zu gewinnen und einen Beitrag zur Diskussion über die Zukunft der kreativen Gestaltungsprozesse zu leisten.

\subsection*{Aufbau der Arbeit}
Der Aufbau der Arbeit folgt einer logischen Struktur, die es dem Leser ermöglicht, die Entwicklung des Themas nachzuvollziehen. Nach einer einführenden Einleitung werden in Kapitel II die Grundlagen des generativen Designs erläutert, um ein solides Fundament für das weitere Verständnis zu schaffen. Kapitel III widmet sich den verschiedenen Methoden des generativen Designs und gibt einen Überblick über ihre Funktionsweise.

Kapitel IV beschäftigt sich mit den konkreten Anwendungen des generativen Designs in verschiedenen Bereichen, wobei für jeden Bereich Fallbeispiele präsentiert werden, um die praktische Anwendung zu veranschaulichen. In Kapitel V werden die Herausforderungen und Zukunftsaussichten des generativen Designs diskutiert, wobei ethische, rechtliche und technologische Aspekte betrachtet werden.

Abschließend erfolgt in Kapitel VI eine Zusammenfassung der Ergebnisse, die Beantwortung der Forschungsfrage sowie eine kritische Bewertung und ein Ausblick auf zukünftige Entwicklungen. Die Arbeit wird mit einem Literaturverzeichnis abgeschlossen, das die verwendeten Quellen angibt.




