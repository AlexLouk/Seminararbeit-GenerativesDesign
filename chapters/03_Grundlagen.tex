\subsection*{Definition des Generative Designs}
Generatives Design ist ein innovativer Ansatz, bei dem Algorithmen und computergestützte Methoden in den Gestaltungsprozess integriert werden. Es ermöglicht Designern, mithilfe vordefinierter Regeln und Parametern automatisch Variationen und Iterationen von Designs zu generieren. Im Zentrum steht die Idee, den Computer als kreativen Partner einzubeziehen, um komplexe und innovative Lösungen zu entwickeln, die über traditionelle manuelle oder konventionelle Ansätze hinausgehen.

Die Grundlagen des Generativen Designs umfassen mathematische Modelle, Regelsysteme und Algorithmen, die formale und ästhetische Eigenschaften von Designs beschreiben und manipulieren können. Es ermöglicht hohe Flexibilität und Anpassungsfähigkeit, da Designs schnell und effizient erstellt, angepasst und optimiert werden können. Generatives Design bietet Designern die Möglichkeit, verschiedene Variationen und Optionen zu erkunden, um neue Perspektiven zu gewinnen und innovative Lösungen zu entwickeln.
\subsection*{Historischer Überblick}

Der historische Überblick des Generativen Designs reicht bis in die 1960er und 1970er Jahre zurück, als erste Experimente mit computergestützter Gestaltung durchgeführt wurden. Zu dieser Zeit begannen Designer und Forscher, den Einsatz von Algorithmen und computergestützten Methoden zu erkunden, um kreative Prozesse zu unterstützen.

In den folgenden Jahrzehnten wurden erhebliche Fortschritte in der Computertechnologie und der Algorithmik erzielt, was zu einer breiteren Anwendung generativer Designmethoden führte. Insbesondere mit dem Aufkommen leistungsfähiger Computer und der Entwicklung spezialisierter Designsoftware wurde das Potenzial des Generativen Designs weiter ausgeschöpft.

Heutzutage ist generatives Design in verschiedenen Bereichen der Gestaltung weit verbreitet. Es findet Anwendung in der Architektur, Produktgestaltung, Grafikdesign und Kunst, Modedesign sowie im Industriedesign. Dabei werden spezifische generative Designmethoden verwendet, um die jeweiligen Anforderungen und Herausforderungen in den einzelnen Bereichen zu bewältigen.

